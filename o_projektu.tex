\documentclass{beamer}

\usepackage[utf8]{inputenc}
\usepackage{default}
\usepackage[croatian]{babel}
\usepackage{pseudocode}
\usetheme{Boadilla}

\title[3-SAT and MAX-SAT]{Ispunjivost logičkih formula}
\subtitle{3SAT i MAX-SAT}
\author[A. Čabraja,A. Štajduhar,A. Fofonjka]{Anto \v{C}abraja, Andrija \v{S}tajduhar, Anamarija Fofonjka}
\institute[PMF-MO]{
  Računarstvo i matematika\\
  Sveučilište u Zagrebu, Prirodoslovno matematički fakultet\\
  Bijenička 30, Zagreb\\[1ex]
  \texttt{cabraja.anto@gmail.com}\\
  \texttt{astajd@gmail.com}\\
  \texttt{fofonjka@math.hr}
}

\begin{document}

\section{Naslov}
\begin{frame}
  \titlepage
\end{frame}

\section{Opis problema}
\begin{frame}
 \frametitle{Ispunjivost logičkih formule}
 Provjera da li za zadanu formulu postoji interpretacija tako da je formula istinita\\
 Specijalni oblici \emph{konjunktivna normalna forma} i \emph{disjunktivna normalna forma}\\
 Problem ispunjivosti logičkih formula obično se algoritamski provjerava na konjunktivnim normalnim formama\\
 Postoje teoremi koji nam jamče da za svaku formulu postoji KNF i DNF i nisu jedinstveni
\end{frame}

\begin{frame}
 \frametitle{Problem koji proučavamo}
 \begin{itemize}
  \item 3-SAT
  \item MAX-SAT
  \item MAX-3-SAT
 \end{itemize}

\end{frame}


\subsection{3-SAT}
\begin{frame}
 \frametitle{3-SAT opis:}
 Formule u konjunktivnoj normalnoj formi sa po 3 literala u svakoj elementarnoj konjunkciji npr:\\
 $(P_1 \vee P_2 \vee P_3) \wedge (\neg P_2 \vee P_3 \vee P_4) .....$ \\
 Dokazano da je NP-teško \\
 Znaimljivo da za 1-SAT i 2-SAT postoje polinomijalni algoritmi koji na determinističkom Turingovom stroju daju rješenje
\end{frame}

\begin{frame}
 \frametitle{3-SAT načini realizacije:Paraleno}
 Paradigme:\\
 \begin{itemize}
  \item Genetski algoritam 
  \item Slučajne šetnje
  \item Simulirano kaljenje i/ili Hill climbing
 \end{itemize}
\end{frame}

\begin{frame}
 \frametitle{Motivacija i primjena}
 \textbf{motivacija:}\\
 Već dugo poznat problem, mogućnost paralelizacije korištenjem više paradigmi, jaka matematička podloga\\
 \textbf{primjena:}
\end{frame}
\subsection{MAX-SAT}
\begin{frame}
 \frametitle{MAX-SAT opis:}
\end{frame}

\begin{frame}
 \frametitle{MAX-SAT realizacija:}
 Paralelno\\
 Paradigme:
 \begin{itemize}
  \item Genetski algoritam
  \item ja bi pokušao kombinaciju još nekog al to ti odluči koji
 \end{itemize}
\end{frame}
 

\begin{frame}
 \frametitle{Motivacija i primjene}
\end{frame}

\subsection{MAX-3-SAT}
\begin{frame}
 \frametitle{MAX-3-SAT}
 Veza između promatranih problema \\
 Realizacija korištenjem već gotovih algoritama za navedene probleme 3-SAT i MAX-SAT
\end{frame}

\section{Tehnologije i zaključak}
\begin{frame}
 \frametitle{Tehnologije i podrška za realizaciju projekta}
 Izrada u C/C++ programskom jeziku uz korištenje standardnih biblioteka \\
 Kompajler gcc i g++\\
 Tjek izrade projekta moguće pratiti na \url{ github.com/acabraja/meta-projekt}
\end{frame}

\begin{frame}
 \begin{center}
  \textbf{Hvala na pažnji!}
 \end{center}
\end{frame}

\end{document}
